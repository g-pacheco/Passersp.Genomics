\thispagestyle{plain} % Page style without header and footer
\chapter*{Sobre o Proponente}

\hspace*{2em}Nascido em Natal a 25 de março de 1989 de mãe cearense e pai potiguar, \textbf{George Pacheco} foi criado na Cidade do Sol onde se graduou bacharel e licenciado em Ciências Biológicas pela \textit{\href{https://www.ufrn.br/}{Universidade Federal do Rio Grande}} no ano de 2011. Tendo recebido um financiamento governamental dinamarquês para cursar o programa de mestrado em Biologia da \textit{\href{https://www.ku.dk/english/}{Universidade de Copenhague}}, deixa a sua cidade natal já no início de 2012. Para a sua tese de mestrado, trabalha com dados genômicos do pombo-das-rochas sob uma perspectiva evolutiva, e defende sua dissertação de mestrado em meados de 2014. Permanece em Copenhague para tirar o título de doutor pela mesma instituição através do programa do governo brasileiro \textit{Ciência Sem Fronteiras} ainda trabalhando na genômica evolutiva do pombo-das-rochas. Após defender o teu título de doutorado, \textbf{George} conseguiu um financiamento para visitar por três meses o laboratório de \href{https://www.cibio.up.pt/en/people/details/miguel-jorge-pinto-carneiro/}{Miguel Carneiro} no centro de pesquisa lusitano \href{https://www.cibio.up.pt/en/}{CIBIO}. Tendo retornado à capital dinamarquesa, o proponente primeiro excerceu o cargo de pós-doc no mesmo grupo do seu mestrado e doutorado, e depois também foi pós-doc por 20 meses na \textit{\href{https://www.dtu.dk/english/}{Universidade Técnica da Dinamarca}} no interior do país. Hoje aos 35 anos \textbf{George} exerce um terceiro pós-doc pela \textit{\href{https://www.uio.no/english/}{Universidade de Oslo}}, no qual continua trabalhando com genômica evolutiva de aves, sendo desta vez dos pardais. \\
\hspace*{2em}Embora tenha estado fora do Brasil desde seu mestrado, \textbf{George} tem mantido uma boa relação com outros cientistas brasileiros. Isto pode ser atestado pelo fato que o proponente é co-autor de cinco publicações de autores brasileiros resultantes de projetos realizados já durante o seu período no exterior. Ademais, depois de sua participação no Encontro da Sociedade Portuguesa de Biologia Evolutiva em 2019, \textbf{George} concebeu a fundação da \textbf{\href{https://sbbevol.org/estatuto/}{Sociedade Brasileira de Biologia Evolutiva}} (\textbf{\href{https://sbbevol.org/estatuto/}{SBBE}}). Assim, \textit{juntamente com vários outros biólogos evolutivos brasileiros}, \textbf{George} tem trabalhado arduamente tanto na criação da \textbf{\href{https://sbbevol.org/estatuto/}{SBBE}} como na realização do \textbf{\href{https://sbbevol.org/}{I Congresso Brasileiro de Biologia Evolutiva}} (\textbf{\href{https://sbbevol.org/}{SBBE24}}).

\vfill

\begin{importantbox}
\raggedleft
{\large\textit{Donde concluo que um dos ofícios do homem é fechar e apertar muito os olhos, e ver se continua pela noite velha o sonho truncado na noite moça.}} \\
\textbf{Dom Casmurro} • Machado de Assis
\end{importantbox}