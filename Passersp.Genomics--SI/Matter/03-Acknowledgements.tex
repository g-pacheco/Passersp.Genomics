\thispagestyle{plain}
\chapter*{Agradecimentos}

\hspace*{2em}Primeiramente, sou imensamente grato ao meu pai \textbf{Jorge Pacheco} e ao meu tio \textbf{Professor Paulo Pacheco} por terem imbuído em mim quando menino o desejo de estudar a nossa fauna e flora local. Sendo eu nascido e criado na capital potiguar, sempre ouvi com deleite as estórias de suas andanças pelos arredores de Lajes do Cabuji --- no interior do Rio Grande do Norte --- onde ficava o pedaço de terra dos meus bisavós. A briga do tijuaçu contra a cobra perigosa; a arrevoada da ribaçã; o canto do golinha e da juriti. Todas essas memórias me fazem escrever a proposta que aqui apresento de uma maneira que sinto que me é própria; que me pertence. \\
\hspace*{2em}Também sou muitíssimo grato ao meu amigo \textbf{\href{https://scholar.google.fr/citations?user=gvZmPNQAAAAJ&hl=fr}{Filipe Vieira}}, com quem eu discuti as ideias presentes nesta proposta durante os anos em que fomos colegas no \textit{\href{https://snm.dk/en}{Museu de História Natural da Dinamarca}}, em Copenhague. Devo confessar que \textbf{Filipe} sempre foi, e ainda é, um descrente da coisa como um todo. No entanto, isto nunca o impediu de tentar me ajudar a melhor lapidar os meus pensamentos. \\
\hspace*{2em}Devo também gratidão a dois amigos da Cidade das Dunas: \textbf{\href{https://scholar.google.com.br/citations?user=NQQBxawAAAAJ&hl=pt-BR}{Paulo Leonardo}} \& \textbf{\href{https://waldirmbf.github.io}{Waldir Miron}}. Ambos já ouviram, reouviram e triouviram diversas versões das ideias expostas nesta proposta, sem jamais deixarem de ter algo a dizer sobre elas, cada um em sua respectiva área. Ademais, estes dois amigos, juntamente com um terceiro, de nome \textbf{\href{https://scholar.google.com/citations?user=X-JyQfoAAAAJ&hl=en}{Rosemberg Menezes}}, sempre me incentivaram a considerar o cenário no qual eu voltaria a fazer ciência em ares potiguares. \\
\hspace*{2em}Por fim, muito agradeço a \textbf{\href{https://www.instagram.com/will_pessoa_bio}{Will Pessoa}}, um dos verdadeiros entendedores da questão, por ter me permitido usar a sua lindíssima foto para ilustrar a capa deste texto e ainda por ter compartilhado comigo, durante as últimas semanas, um pouco do seu vasto conhecimento sobre a nossa amada biodiversidade natal.
 
\vfill

\begin{importantbox}
\raggedleft
{\Small{\textbf{Pertencer} \\
\textit{Pertencer não vem apenas de ser fraca e precisar unir-se a algo ou a alguém mais forte. Muitas vezes a vontade intensa de pertencer vem em mim de minha própria força - eu quero pertencer para que minha força não seja inútil e fortifique uma pessoa ou uma coisa.}} \\
\vspace{1em}
\textbf{Se eu fosse eu} \\
\textit{No entanto já li biografias de pessoas que de repente passavam a ser elas mesmas, e mudavam inteiramente de vida. Acho que se eu fosse realmente eu, os amigos não me cumprimentariam na rua, porque até minha fisionomia teria mudado. Como? Não sei.}} \\
\textbf{A Descoberta do Mundo} • Clarice Lispector
\end{importantbox}